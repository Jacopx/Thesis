\documentclass[%
    corpo=12pt,
    twoside,
%    stile=classica,
    oldstyle,
    autoretitolo,
    greek,
    evenboxes,
%    tipotesi,
]{toptesi}
%%%%%%%%%%%%%%%%%%%%%%%%%%%%%%%%%%%%%%%%%%%%%%%%%%%%

\usepackage[utf8]{inputenc}
\usepackage[T1]{fontenc}
\usepackage{lmodern}
\usepackage{hyperref}
\usepackage{graphicx}
\usepackage{subfigure}
\usepackage{booktabs}
\usepackage{amsfonts}
\usepackage{amssymb}
\usepackage{bm}
\usepackage{listings}

% \hypersetup{%
%     pdfpagemode={UseOutlines},
%     bookmarksopen,
%     pdfstartview={FitH},
%     colorlinks,
%     linkcolor={blue},
%     citecolor={blue},
%     urlcolor={blue}
%   }

%%%%%%% Definizioni locali
\newtheorem{osservazione}{Osservazione}% Standard LaTeX


\begin{document}

\ateneo{Politecnico di Torino}
%
% Non tutte le università hanno un nome proprio
%\nomeateneo{Sede di Torre Elettra}
%
% \FacoltaDi{Faculty of Computer Engineering}% lo spazio finale correttamente sparisce

\titolo{Time  prediction of software development via machine learning}% per la laurea quinquennale e il dottorato
\sottotitolo{Artificial intelligence applied to Software Engineering}% per la laurea quinquennale e il dottorato
%
%%%%%%% Corso degli studi
\corsodilaurea{Computer Engineering}% per la laurea
%\corsodidottorato{Meccanica}% per il dottorato

\renewcommand*\IDlabel{}
%
\candidato{Jacopo \textsc{Nasi} [255320]}

%%%%%%% Relatori o supervisori
\relatore{prof.~Maurizio Morisio}

%%%%%%% Tutore
\tutoreaziendale{dott.\ Davide Piagneri}
\NomeTutoreAziendale{Supervisore aziendale\\EisWORLD SRL}

\sedutadilaurea{\textsc{Anno~accademico} 2019-2020}


%%%%%%% Logo della sede
\logosede{polito}

%%%%%%% Per cambiare l'offset per la rilegatura; meno offset
%%%%%%% c'e', meglio e'
%\setbindingcorrection{3mm}

\english%  di default e' in vigore \italiano

\iflanguage{english}{%
	\retrofrontespizio{This work is subject to the Creative Commons Licence}
	\DottoratoIn{PhD Course in\space}
	\CorsoDiLaureaIn{Master degree course in\space}
	\NomeMonografia{Bachelor Degree Final Work}
	\TesiDiLaurea{Master Degree Thesis}
	\NomeDissertazione{PhD Dissertation}
	\InName{in}
	\CandidateName{Candidates}% or Candidate
	\AdvisorName{Supervisors}% or Supervisor
	\TutorName{Tutor}
	\NomeTutoreAziendale{Internship Tutor}
	\CycleName{cycle}
	\NomePrimoTomo{First volume}
	\NomeSecondoTomo{Second Volume}
	\NomeTerzoTomo{Third Volume}
	\NomeQuartoTomo{Fourth Volume}
	\logosede{polito}% or comma separated list of logos
}{}
%%%%%%%%%%%%%%%%%%%%%%%%%%%%%%%%%%%%%%%%%

\frontespizio

\summary

La pressione barometrica di Giove viene misurata
mediante un metodo originale  messo a punto dai candidati, che si basa
sul rilevamento telescopico della pressione.

% \paginavuota % funziona anche senza specificare l'opzione classica

\acknowledgements

I candidati ringraziano vivamente il Granduca di Toscana per i mezzi
messi loro a disposizione, ed il signor Von Braun, assistente del
prof.~Albert Einstein, per le informazioni riservate che egli ha
gentilmente fornito loro, e per le utili discussioni che hanno permesso
ai candidati di evitare di riscoprire l'acqua calda.


\indici

\mainmatter

\part{Prima Parte}
\chapter{Introduzione generale}

\section{Principi generali}
Il problema della determinazione della pressione barometrica dell'atmosfera di
Giove non ha ricevuto finora una soluzione soddisfacente, per l'elementare
motivo che il pianeta suddetto si trova ad una distanza tale che i mezzi attuali
non consentono di eseguire una misura diretta. \cite{adams2}


\bibliography{biblio}
\bibliographystyle{QUICKtran}

\end{document}
