% * * * * * * * * * * * * * * * * * * * * * * * * * * * * * * * * * * *
% *                             Thesis                                *
% *                 https://github.com/Jacopx/Thesis                  *
% * * * * * * * * * * * * * * * * * * * * * * * * * * * * * * * * * * *

\documentclass[%
    corpo=12pt,
    twoside,
%    stile=classica,
    oldstyle,
    autoretitolo,
    greek,
    evenboxes,
%    tipotesi,
]{toptesi}
%%%%%%%%%%%%%%%%%%%%%%%%%%%%%%%%%%%%%%%%%%%%%%%%%%%%

\usepackage[utf8]{inputenc}
\usepackage[T1]{fontenc}
\usepackage{lmodern}
\usepackage{hyperref}
\usepackage{graphicx}
\usepackage{subfigure}
\usepackage{booktabs}
\usepackage{amsfonts}
\usepackage{amssymb}
\usepackage{bm}
\usepackage{listings}

\hypersetup{%
    pdfpagemode={UseOutlines},
    bookmarksopen,
    pdfstartview={FitH},
    colorlinks,
    linkcolor={blue},
    citecolor={blue},
    urlcolor={blue}
  }

%%%%%%% Definizioni locali
\newtheorem{osservazione}{Osservazione}% Standard LaTeX


\begin{document}

\ateneo{Politecnico di Torino}
%
% Non tutte le università hanno un nome proprio
%\nomeateneo{Sede di Torre Elettra}
%
% \FacoltaDi{Faculty of Computer Engineering}% lo spazio finale correttamente sparisce

\titolo{Time  prediction of software development via machine learning}% per la laurea quinquennale e il dottorato
\sottotitolo{Artificial Intelligence applied to Software Engineering}% per la laurea quinquennale e il dottorato
%
%%%%%%% Corso degli studi
\corsodilaurea{Computer Engineering}% per la laurea
%\corsodidottorato{Meccanica}% per il dottorato

\renewcommand*\IDlabel{}
%
\candidato{Jacopo \textsc{Nasi} [255320]}

%%%%%%% Relatori o supervisori
\relatore{prof.~Maurizio Morisio}

%%%%%%% Tutore
\tutoreaziendale{dott.\ Davide Piagneri}
\NomeTutoreAziendale{Supervisore aziendale\\EisWORLD SRL}

\sedutadilaurea{\textsc{Anno~accademico} 2019-2020}


%%%%%%% Logo della sede
\logosede{polito}

%%%%%%% OFFSET
%\setbindingcorrection{3mm}

\english%  di default e' in vigore \italiano

\iflanguage{english}{%
	\retrofrontespizio{This work is subject to the Creative Commons Licence}
	\DottoratoIn{PhD Course in\space}
	\CorsoDiLaureaIn{Master degree course in\space}
	\NomeMonografia{Bachelor Degree Final Work}
	\TesiDiLaurea{Master Degree Thesis}
	\NomeDissertazione{PhD Dissertation}
	\InName{in}
	\CandidateName{Candidates}% or Candidate
	\AdvisorName{Supervisors}% or Supervisor
	\TutorName{Tutor}
	\NomeTutoreAziendale{Internship Tutor}
	\CycleName{cycle}
	\NomePrimoTomo{First volume}
	\NomeSecondoTomo{Second Volume}
	\NomeTerzoTomo{Third Volume}
	\NomeQuartoTomo{Fourth Volume}
	\logosede{polito}% or comma separated list of logos
}{}
%%%%%%%%%%%%%%%%%%%%%%%%%%%%%%%%%%%%%%%%%

\frontespizio

\summary

La pressione barometrica di Giove viene misurata
mediante un metodo originale  messo a punto dai candidati, che si basa
sul rilevamento telescopico della pressione.

% \paginavuota % funziona anche senza specificare l'opzione classica

\acknowledgements

Un ringraziamento speciale ai cavalieri di Smirnuff, luce della mia battaglia.

\indici

\mainmatter

% #######################################
% #            FIRST CHAPTER            #
% #######################################

\chapter{Introduction}

\section{General Problem}
Lorem ipsum dolor sit amet, consectetur adipisicing elit, sed do eiusmod tempor incididunt ut labore et dolore magna aliqua. Ut enim ad minim veniam, quis nostrud exercitation ullamco laboris nisi ut aliquip ex ea commodo consequat. Duis aute irure dolor in reprehenderit in voluptate velit esse cillum dolore eu fugiat nulla pariatur. Excepteur sint occaecat cupidatat non proident, sunt in culpa qui officia deserunt mollit anim id est laborum. \cite{adams2}

\section{Tools used}
Lorem ipsum dolor sit amet, consectetur adipisicing elit, sed do eiusmod tempor incididunt ut labore et dolore magna aliqua. Ut enim ad minim veniam, quis nostrud exercitation ullamco laboris nisi ut aliquip ex ea commodo consequat. Duis aute irure dolor in reprehenderit in voluptate velit esse cillum dolore eu fugiat nulla pariatur. Excepteur sint occaecat cupidatat non proident, sunt in culpa qui officia deserunt mollit anim id est laborum.\cite{dummy3}

% #######################################
% #           SECOND CHAPTER            #
% #######################################

\chapter{Datasets}
\section{SEOSS33}
\section{San Francisco Bike Sharing}
\section{San Francisco Fire Department}


% #######################################
% #           THIRD CHAPTER            #
% #######################################

\chapter{Machine Learning models}
\section{Random Forest}
\section{Neural Networks}


% #######################################
% #           FOURTH CHAPTER            #
% #######################################

\chapter{Forecasting}
\section{Goal introduction}
\section{Feature extraction}
\section{Models detail}
\section{Forecasting horizons}
\section{Feature application}
\section{Application over different projects}


% #######################################
% #           FIFTH CHAPTER            #
% #######################################

\chapter{Model abstraction}
\section{CommonDB}
\section{SFBS and literature comparisons}
\section{SFFD}


% #######################################
% #           SIXTH CHAPTER             #
% #######################################

\chapter{Conclusion}



% #######################################
% #            BIBLIOGRAPHY             #
% #######################################
\bibliography{biblio}
\bibliographystyle{QUICKtran}

\end{document}
